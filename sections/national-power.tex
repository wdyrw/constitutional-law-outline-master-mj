\section{The Contemporary Debate over National Power}

\subsection{Federalism: Limits on the Commerce Clause}

\subsubsection{The 1960s Civil Rights Legislation: Commerce or Reconstruction? 
(\emph{Heart of Atlanta Motel} and \emph{McClung})}

\begin{enumerate}
    \item ``Second Reconstruction'': the result of the civil rights movement 
    in the 50s and 60s.\footnote{Casebook p. 558.}
    \item A central question was whether the federal government's authority to 
    enact civil rights legislation rested on the Commerce Clause or \S\ 2 
    of the Thirteenth Amendment\footnote{``Congress shall have the power to 
    enforce this article by appropriate legislation.''} and \S\ 5 of the 
    Fourteenth Amendment.\footnote{``The Congress shall have the power to 
    enforce, by appropriate legislation, the provisions of this article.''}
    \item Many questioned the legitimacy of basing social legislation on the 
    Commerce Clause, but Congress ultimately chose that route.
    \item Two cases challenged the legitimacy of basing civil rights 
    legislation on the Commerce Clause:
    \begin{enumerate}
        \item \emph{Heart of Atlanta Motel v. United States}: a Georgia motel 
        challenged Title II. The Court unanimously upheld the statute, holding 
        that the motel ``stood readily accessible to interstate highways, 
        advertised in various national media, and served a clientele 75 
        percent of which came from out of state.''\footnote{Casebook p. 560.} 
        Congress can enact moral regulation on the basis of its Commerce 
        Clause power.
        \item \emph{Katzenbach v. McClung}: an Alabama restaurant challenged 
        Title II. The Court held that racial discrimination affects interstate 
        commerce. It also held that rational review is the appropriate 
        standard for Commerce Clause cases.
    \end{enumerate}
    \item Two later cases, \emph{Daniel v. Paul} and \emph{Perez v. United 
    States}, also upheld the Civil Rights Act. But Justice Black dissented in 
    \emph{Daniel}, arguing that the Court's interpretation of the Commerce 
    Clause gave the federal government ``complete control over every little 
    remote country place of recreation in every nook and cranny of every 
    precinct and county in every one of the 50 states.''\footnote{Casebook p. 
    563.} Justice Stewart dissented in \emph{Perez}, arguing that if a class 
    of people as a whole affects interstate commerce, prosecutors should still 
    have to show whether individual members of that class themselves have an 
    effect.
\end{enumerate}

\subsubsection{The Rehnquist Court: Finding Limits on Federal Power}

\begin{enumerate}
    \item \emph{Oregon v. Mitchell} was the only case between 1937 and 1987 to 
    hold an act of Congress unconstitutional based on a lack of enumerated 
    power. The Rehnquist Court signaled a shift towards limiting federal 
    powers.
\end{enumerate}

\subsubsection{\emph{United States v. Lopez}}

\begin{enumerate}
    \item The Gun-Free School Zones Act criminalized possession of guns in 
    school zones. Lopez, a high school student, was caught with a gun at 
    school. He challenged the act as beyond the scope of congressional power 
    under the Commerce Clause.
    \item Justice Rehnquist:
    \begin{enumerate}
        \item Earlier cases (\emph{Jones \& Laughlin Steel}, \emph{Darby}, 
        \emph{Wickard}) established Congress's authority to enact social 
        legislation under its Commerce Clause Authority. They allowed Congress 
        to (1) regulate activity that uses the \emph{channels} of interstate 
        commerce, (2) protect the \emph{instrumentalities} of interstate 
        commerce, and (3) regulate activities with a \emph{substantial 
        relation} to interstate commerce.\footnote{Casebook pp. 601--02.} The 
        first two categories were irrelevant in this case. Only the third is 
        at issue.
        \item The government argued that the statute affects interstate 
        commerce because guns lead to costly violent crime, because gun 
        violence reduces travel and thus impedes commerce, and because gun 
        violence harms education which in turn harms economic productivity.
        \item If the government's theories were correct, would there be any 
        areas of criminal law that the government could not regulate under the 
        Commerce Clause? Any activity can be characterized as 
        commercial.\footnote{Casebook pp. 603--04.}
    \end{enumerate}
    \item Justice Kennedy, concurring:
    \begin{enumerate}
        \item Federalism was one of the framers' key values.
        \item Broadening the Commerce Clause would disrupt the balance of 
        power between state governments and the federal government.
        \item Authority to regulate education lies with the states, and 
        education is beyond commerce. States, however, are free to enact 
        similar legislation.
    \end{enumerate}
    \item Justice Thomas, concurring:
    \begin{enumerate}
        \item ``~.~.~.~the term `commerce' denoted sale and/or transport 
        rather than business generally.''\footnote{Casebook p. 608.}
        \item A more expansive reading of the meaning of ``commerce'' would 
        render many other constitutional clauses superfluous.
        \item ``Taken together, these fundamental textual problems should, at 
        the very least, convince us that the `substantial effects' test should 
        be reexamined~.~.~.''\footnote{Casebook p. 609.}
    \end{enumerate}
    \item Justice Stevens, dissenting:
    \begin{enumerate}
        \item The welfare of commerce depends on education.
        \item ``Guns are both articles of commerce and articles that can be 
        used to restrain commerce. Their possession is the consequence, either 
        directly or indirectly, of commercial activity.''\footnote{Casebook p. 
        610.}
    \end{enumerate}
    \item Justice Souter, dissenting:
    \begin{enumerate}
        \item 1900--1937: the Court invalidated federal social and economic 
        legislation on the basis \enquote{highly formalistic notions of 
        \enquote{commerce}}---e.g., \emph{Hammer v. Dagenhart} (striking down 
        legislation prohibiting commerce in goods manufactured with child 
        labor).\footnote{Casebook p. 611.} After 1937, the Court's Commerce 
        Clause jurisprudence expanded (\emph{Jones \& Laughlin Steel}, 
        \emph{Darby}, \emph{Wickard}, and rational review).
        \item The majority opinion here is ``a backward glance at old 
        pitfalls~.~.~.''\footnote{Casebook p. 612.}
        \item There is a close connection between education and commerce.
        \item Congress's Commerce Clause power is plenary. It doesn't matter 
        whether the power to regulate education lies with the states.
        \item Congress's findings on the effects of gun violence are not 
        determinative.\footnote{Casebook pp. 613--14.}
    \end{enumerate}
    \item Justice Breyer, dissenting:
    \begin{enumerate}
        \item Three principles of Commerce Clause interpretation:
        \begin{enumerate}
            \item Congress can regulate activity that significantly effects 
            interstate commerce. \emph{Wickard.}
            \item Congress can regulate activity on the basis of its 
            cumulative, not separate, effect. \emph{Wickard.}
            \item Rational review is the proper standard.
        \end{enumerate}
        \item There is a substantial connection between this legislation and 
        interstate commerce because gun violence has marked effects on 
        education, and in turn, on commerce.\footnote{Casebook pp. 615--16.}
        \item Upholding this act would not expand Congress's Commerce Clause 
        power.
        \item The majority's distinction between commercial and noncommercial 
        activities is tenuous. And anyway, education has a clear impact on 
        commerce. ``~.~.~.~Congress has often analyzed school expenditure as 
        if it were a commercial investment~.~.~.''\footnote{Casebook pp. 
        618--19.}
    \end{enumerate}
\end{enumerate}
 
\subsubsection{The Constitutionality of Health Care Reform: \emph{National 
Federation of Independent Businesses v. Sibelius}}

\begin{enumerate}
    \item At issue was the individual mandate of the Patient Protection and 
    Affordable Care Act.
    \item Justice Roberts:
    \begin{enumerate}
        \item Federalism guarantees liberty by diffusing sovereign power.
        \item The individual mandate requires people to buy health insurance 
        or pay a penalty to the federal government.
        \item The government argued that the act was valid under (1) the 
        Commerce Clause and (2) Congress's taxing power.
        \item Commerce Clause:
        \begin{enumerate}
            \item Hospitals must treat uninsured patients. They pass on the 
            costs to insurers and insurers pass on the costs to consumers. The 
            individual mandate prevents freeriding.
            \item But the Commerce Clause only gives Congress the power to 
            regulate economic \emph{activity}. It does not grant the power to 
            \emph{create} economic activity.\footnote{Handout p. 5.}
            \item ``Under the government's theory, Congress could address the 
            diet problem by ordering everyone to buy 
            vegetables.''\footnote{Handout p. 6.}
            \item A limitless Commerce Clause would create a Hamiltonian 
            ``impetuous vortex.''\footnote{Handout p. 7.}
            \item All precedent Commerce Clause cases involve the regulation 
            of preexisting economic activity.
        \end{enumerate}
        \item Necessary and Proper Clause:
        \begin{enumerate}
            \item The government argued that the individual mandate should 
            stand because it is an integral part of the broader regulatory 
            scheme.
            \item But the Clause does not permit Congress to exercise great 
            substantive powers outside those enumerated.
            \item The individual mandate may be necessary, but it isn't 
            proper.
        \end{enumerate}
        \item Taxing power:
        \begin{enumerate}
            \item The individual mandate is valid under Congress's taxing 
            power. ``~.~.~.~it makes going without insurance just another 
            thing the government taxes.''\footnote{Handout p. 10.}
        \end{enumerate}
    \end{enumerate}
    \item Justice Ginsburg:
    \begin{enumerate}
        \item The framers intended the Constitution to be a ``great outline'' 
        with the ``capacity to provide for future contingencies as they may 
        happen~.~.~.''\footnote{Handout p. 11, quoting Hamilton in Federalist 
        No. 34.}
        \item Until now, the Court's Commerce Clause analysis asked (1) 
        whether the regulated activities substantially affect interstate 
        commerce and (2) whether there is a rational basis for the legislation 
        and a reasonable connection between the regulation and the goal.
        \item Congress had a rational basis for concluding that the uninsured 
        substantially affected interstate commerce.
        \item Everyone will eventually consume healthcare services. Thus, 
        everyone is involved in interstate commerce of health insurance. This 
        is not inactivity; every activity can be recast as 
        inactivity.\footnote{Handout p. 14.}
        \item The ``broccoli horrible'' is unlikely. Congress had a rational 
        reason to legislate because of the uninsured freeriding problem. Other 
        constitutional provisions could also check Congress's power. Finally, 
        states have always had the power to impose mandates, but they rarely 
        do.
        \item The individual mandate was both necessary and proper.
    \end{enumerate}
    \item Justice Scalia:
    \begin{enumerate}
        \item Abstention from commerce is not commerce.\footnote{Handout p. 
        18.} The individual mandate directs the creation of commerce.
        \item There is not universal participation in the healthcare market, 
        and the federal government has no power to mandate it.
    \end{enumerate}
\end{enumerate}

\subsection{Limits on the Fourteenth Amendment, Section 5}

\subsubsection{Mapping the Middle Ground: \emph{Jones v. Mayer} and 
\emph{Oregon v. Mitchell}}

\begin{enumerate}
    \item % FIXME 591-600
    \item \emph{Jones v. Alfred H. Mayer Co.}: Congress has the power under 
    the Thirteenth Amendment to prohibit racial discrimination in the sale or 
    rental of real estate (upholding the 1866 Civil Rights Act, authorizing 
    Congress to determine ``what are the badges and incidents of slavery, and 
    the authority to translate that determination into effective 
    legislation.'').
\end{enumerate}
 
\subsubsection{The Reconstruction Power}
 
% \begin{enumerate}
%     \item % FIXME 629
% \end{enumerate}

\subsubsection{\emph{City of Boerne v. Flores}}
 
% \begin{enumerate}
%     \item % FIXME 629-49
% \end{enumerate}
 
\subsubsection{\emph{Northwest Austin Municipal Utility District Number One 
(NAMUDNO) v. Holder}}

% \begin{enumerate}
%     \item % FIXME handout from brest 2011 supplement
% \end{enumerate}
