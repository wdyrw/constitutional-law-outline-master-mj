\section{race}

\begin{enumerate}
    \item Sticking five six seven eight strict constitutional norms versus deviations for the 
    sake of exigencies or emergencies---e.g., the Philadelphia Convention's 
    narrow mandate to revise the Articles of Confederation versus the 
    perceived need to draft a new constitution.\footnote{Cf. Casebook p. 21.}
    \item Were Hamilton and Iredells fears about the Bill of Rights---that 
    enumerating limitations on the federal government's powers ``would furnish 
    to men disposed to usurp, a plausible pretence for claiming that 
    power''---vindicated?\footnote{Casebook pp. 25--26.}
    \item \textbf{Constitutionalism}: why does a court have the power to stop 
    ordinary democracy?
    \begin{enumerate}
        \item There have been only two cases where constitutional amendments 
        overruled Supreme Court decisions: Fourteenth (\emph{Dred Scott v. 
        Sandford}) and Twenty-Sixth (\emph{Oregon v. Mitchell}) Amendments.
    \end{enumerate}
    \item Key areas:
    \begin{enumerate}
        \item Constitutionalism.
        \item Federalism.
        \item Liberty.
        \item Equality.
    \end{enumerate}
    \item How does popular legitimacy check government power?
    \begin{enumerate}
        \item Acts at the executive's discretion are ``only politically 
        examinable.'' \emph{Marbury}.
    \end{enumerate}
    \item How does the Supreme Court's power depend on its legitimacy? How 
    does the Court respond to political pressure? What are the implications of 
    the fact that the Court relies on the other branches to enforce its 
    decisions?
    \begin{enumerate}
        \item \emph{National Federation of Independent Business v. Sebelius}.
        \item \emph{Bush v. Gore}.
        \item \emph{Brown v. Board of Education}.
        \item Federalist ``irreconcilables'' in the House ultimately caved and 
        cast their electoral votes for Jefferson.\footnote{Casebook p. 101.}
        \item If the Court's legitimacy depends on the other branches, is it 
        really a countermajoritarian institution?
    \end{enumerate}
    \item Does the Equal Protection Clause require color blindness in all 
    situations? See the history of the Fourteenth Amendment.
    \item Did the Reconstruction amendments only end slavery, or did they 
    establish protections for broader substantive rights?
    \item What to do when the authors of the Constitution or amendments are 
    the same authors of legislation that is in question (e.g., the Judiciary 
    Act of 1789 in \emph{Marbury})?
    \item What's the point of making the Constitution supreme and not just an 
    ordinary piece of legislation?
    \begin{enumerate}
        \item Government is skeptical of itself, so it builds in structural 
        safeguards that cannot be changed without extraordinary effort.
    \end{enumerate}
    \item What types of discrimination does the Fourteenth Amendment allow?
    \begin{enumerate}
        \item At the time of its adoption: denial of suffrage to women and 
        blacks.
    \end{enumerate}
    \item Two views of the role of the judge:
    \begin{enumerate}
        \item Loose, creative interpretation of the law to reach results consistent 
        with morality.
        \item Strict, literal interpretation---``that's just what the law 
        says.''
    \end{enumerate}
    \item Interpretive strategies:
    \begin{enumerate}
        \item Intent---what did the authors mean?
        \item Purpose---what contextual problems did the language address?
        \item Structure---how does the internal structure indicate meaning?
        \item Consequentialism---what would be the consequences of a 
        particular interpretation? (Courts rarely acknowledge that they're 
        doing this, but sometimes they do---e.g., casebook pp. 324--25.)
        \item Natural law---e.g., \emph{Bradwell}.
        \item Settled practices---e.g., \emph{Minor}.\footnote{Casebook p. 345 
        top.}
    \end{enumerate}
    \item Natural law tends to reinforce the status quo, favoring those in 
    power. But it can also be a weapon for challenging the status quo.
    \item Why do courts rely on natural law in some cases and not in others?
    \begin{enumerate}
        \item Natural law is contestable. It's more persuasive to use another 
        interpretive strategy if you have it.
    \end{enumerate}
    \item Regarding the prisoner's dilemma of state economic policy: ``To what 
    extent, if any, should these kinds of game theoretic analyses play a role 
    in the best interpretation of constitutional provisions? Is this simply 
    another example of how consequences are as important a modality of 
    constitutional interpretation as text, history, or original 
    intention?''\footnote{Casebook p. 447.}
    \item To what extent is the Supreme Court an actor in the political 
    system?
    \item Race: anticlassification vs. antisubordination.
    \item Is the due process clause the appropriate place for protecting 
    substantive rights?\footnote{See Chemerinsky p. 559.}
\end{enumerate}

